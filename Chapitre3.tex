% Options for packages loaded elsewhere
\PassOptionsToPackage{unicode}{hyperref}
\PassOptionsToPackage{hyphens}{url}
%
\documentclass[
]{article}
\usepackage{lmodern}
\usepackage{amssymb,amsmath}
\usepackage{ifxetex,ifluatex}
\ifnum 0\ifxetex 1\fi\ifluatex 1\fi=0 % if pdftex
  \usepackage[T1]{fontenc}
  \usepackage[utf8]{inputenc}
  \usepackage{textcomp} % provide euro and other symbols
\else % if luatex or xetex
  \usepackage{unicode-math}
  \defaultfontfeatures{Scale=MatchLowercase}
  \defaultfontfeatures[\rmfamily]{Ligatures=TeX,Scale=1}
\fi
% Use upquote if available, for straight quotes in verbatim environments
\IfFileExists{upquote.sty}{\usepackage{upquote}}{}
\IfFileExists{microtype.sty}{% use microtype if available
  \usepackage[]{microtype}
  \UseMicrotypeSet[protrusion]{basicmath} % disable protrusion for tt fonts
}{}
\makeatletter
\@ifundefined{KOMAClassName}{% if non-KOMA class
  \IfFileExists{parskip.sty}{%
    \usepackage{parskip}
  }{% else
    \setlength{\parindent}{0pt}
    \setlength{\parskip}{6pt plus 2pt minus 1pt}}
}{% if KOMA class
  \KOMAoptions{parskip=half}}
\makeatother
\usepackage{xcolor}
\IfFileExists{xurl.sty}{\usepackage{xurl}}{} % add URL line breaks if available
\IfFileExists{bookmark.sty}{\usepackage{bookmark}}{\usepackage{hyperref}}
\hypersetup{
  pdftitle={Linear Regression},
  pdfauthor={Cheikh Mbacké BEYE},
  hidelinks,
  pdfcreator={LaTeX via pandoc}}
\urlstyle{same} % disable monospaced font for URLs
\usepackage[margin=1in]{geometry}
\usepackage{color}
\usepackage{fancyvrb}
\newcommand{\VerbBar}{|}
\newcommand{\VERB}{\Verb[commandchars=\\\{\}]}
\DefineVerbatimEnvironment{Highlighting}{Verbatim}{commandchars=\\\{\}}
% Add ',fontsize=\small' for more characters per line
\usepackage{framed}
\definecolor{shadecolor}{RGB}{248,248,248}
\newenvironment{Shaded}{\begin{snugshade}}{\end{snugshade}}
\newcommand{\AlertTok}[1]{\textcolor[rgb]{0.94,0.16,0.16}{#1}}
\newcommand{\AnnotationTok}[1]{\textcolor[rgb]{0.56,0.35,0.01}{\textbf{\textit{#1}}}}
\newcommand{\AttributeTok}[1]{\textcolor[rgb]{0.77,0.63,0.00}{#1}}
\newcommand{\BaseNTok}[1]{\textcolor[rgb]{0.00,0.00,0.81}{#1}}
\newcommand{\BuiltInTok}[1]{#1}
\newcommand{\CharTok}[1]{\textcolor[rgb]{0.31,0.60,0.02}{#1}}
\newcommand{\CommentTok}[1]{\textcolor[rgb]{0.56,0.35,0.01}{\textit{#1}}}
\newcommand{\CommentVarTok}[1]{\textcolor[rgb]{0.56,0.35,0.01}{\textbf{\textit{#1}}}}
\newcommand{\ConstantTok}[1]{\textcolor[rgb]{0.00,0.00,0.00}{#1}}
\newcommand{\ControlFlowTok}[1]{\textcolor[rgb]{0.13,0.29,0.53}{\textbf{#1}}}
\newcommand{\DataTypeTok}[1]{\textcolor[rgb]{0.13,0.29,0.53}{#1}}
\newcommand{\DecValTok}[1]{\textcolor[rgb]{0.00,0.00,0.81}{#1}}
\newcommand{\DocumentationTok}[1]{\textcolor[rgb]{0.56,0.35,0.01}{\textbf{\textit{#1}}}}
\newcommand{\ErrorTok}[1]{\textcolor[rgb]{0.64,0.00,0.00}{\textbf{#1}}}
\newcommand{\ExtensionTok}[1]{#1}
\newcommand{\FloatTok}[1]{\textcolor[rgb]{0.00,0.00,0.81}{#1}}
\newcommand{\FunctionTok}[1]{\textcolor[rgb]{0.00,0.00,0.00}{#1}}
\newcommand{\ImportTok}[1]{#1}
\newcommand{\InformationTok}[1]{\textcolor[rgb]{0.56,0.35,0.01}{\textbf{\textit{#1}}}}
\newcommand{\KeywordTok}[1]{\textcolor[rgb]{0.13,0.29,0.53}{\textbf{#1}}}
\newcommand{\NormalTok}[1]{#1}
\newcommand{\OperatorTok}[1]{\textcolor[rgb]{0.81,0.36,0.00}{\textbf{#1}}}
\newcommand{\OtherTok}[1]{\textcolor[rgb]{0.56,0.35,0.01}{#1}}
\newcommand{\PreprocessorTok}[1]{\textcolor[rgb]{0.56,0.35,0.01}{\textit{#1}}}
\newcommand{\RegionMarkerTok}[1]{#1}
\newcommand{\SpecialCharTok}[1]{\textcolor[rgb]{0.00,0.00,0.00}{#1}}
\newcommand{\SpecialStringTok}[1]{\textcolor[rgb]{0.31,0.60,0.02}{#1}}
\newcommand{\StringTok}[1]{\textcolor[rgb]{0.31,0.60,0.02}{#1}}
\newcommand{\VariableTok}[1]{\textcolor[rgb]{0.00,0.00,0.00}{#1}}
\newcommand{\VerbatimStringTok}[1]{\textcolor[rgb]{0.31,0.60,0.02}{#1}}
\newcommand{\WarningTok}[1]{\textcolor[rgb]{0.56,0.35,0.01}{\textbf{\textit{#1}}}}
\usepackage{graphicx}
\makeatletter
\def\maxwidth{\ifdim\Gin@nat@width>\linewidth\linewidth\else\Gin@nat@width\fi}
\def\maxheight{\ifdim\Gin@nat@height>\textheight\textheight\else\Gin@nat@height\fi}
\makeatother
% Scale images if necessary, so that they will not overflow the page
% margins by default, and it is still possible to overwrite the defaults
% using explicit options in \includegraphics[width, height, ...]{}
\setkeys{Gin}{width=\maxwidth,height=\maxheight,keepaspectratio}
% Set default figure placement to htbp
\makeatletter
\def\fps@figure{htbp}
\makeatother
\setlength{\emergencystretch}{3em} % prevent overfull lines
\providecommand{\tightlist}{%
  \setlength{\itemsep}{0pt}\setlength{\parskip}{0pt}}
\setcounter{secnumdepth}{-\maxdimen} % remove section numbering

\title{Linear Regression}
\author{Cheikh Mbacké BEYE}
\date{2020-08-29}

\begin{document}
\maketitle

{
\setcounter{tocdepth}{2}
\tableofcontents
}
\hypertarget{environnement-de-travail}{%
\subsection{ENVIRONNEMENT DE TRAVAIL}\label{environnement-de-travail}}

\begin{Shaded}
\begin{Highlighting}[]
\KeywordTok{library}\NormalTok{(dplyr)}
\KeywordTok{library}\NormalTok{(tidyr)}
\KeywordTok{library}\NormalTok{(ggplot2)}
\end{Highlighting}
\end{Shaded}

Linear regression is a very simple approach for supervised learning. In
particular, linear regression is useful tool for predicting a
quantitative response. The data set used in this part is the
\emph{Advertising} data. It describes sales( in thousand of units) for a
particular product as function of advertising budgets (in thousand of
dollars) for \emph{TV}, \emph{Radio} and \emph{Newspaper} media. Suppose
that our role as data scientist is the answer the following questions
based in \emph{Advertising} data.

\begin{itemize}
\item
  1 is there a relationship between advertising budget and sales? Our
  first goal should be to determine whether the data provide evidence of
  an association between a devrtising expenditure and sales. If the
  evidence is weak, the one might argue that no money should be spent on
  adevtising.\\
\item
  2 How stong is the relationship between advertising budget and sales?
  Assuming that there is a relationship between aadvertising budget and
  sales, we would like to know how strong is the relationship. In other
  words, given a certaiin advertising budget, can we predict sales
  accurately?
\item
  3 Which media contibutes to sales? Do all three media: \emph{TV},
  \emph{Radio} and \emph{Newspaper} contribute to sales or do just one
  or two of the media contribute? To answer the question, we must find a
  way to separate out the invidual effects of each medium when we have
  spent money on all three media.
\item
  4 How accurately can we estimate the effect of each of medium on
  sales? For every dollar spent on advertising in a particular medium,
  by what amount will sales increase? How accurately can we predict this
  amount of increase?
\item
  5 How accurately can we predict future sales? For any given level of
  television, radio or newspaper advertising, what is our prediction for
  sales and what is the accuracy of this prediction?
\item
  6 Is the ralation linear ? If there is approximatively a straight-line
  relationship between adevertising expenditure in the various media and
  sales, then linear regression is an appropriate tool. If not, then it
  may still possible to transform the predictor or the response so that
  linear regression can be used.
\item
  7 Is there a synergy among the advertising media? Perhaps spending
  \$50,000 on television advertising and \$50,000 on radio advertising
  results in more sales than allocating \$100,000 to either tekevision
  or radio individualy. This situation is known as \emph{interaction}
  effect.
\end{itemize}

\hypertarget{simple-linear-regression}{%
\subsection{SIMPLE LINEAR REGRESSION}\label{simple-linear-regression}}

Simple regression lives up its name: it's a very straightforward
approach for predicting a quantitative response \emph{Y} on the basis of
a single predictor variable \emph{X}. It assumes that there is a linear
relationship between \emph{X} and \emph{Y}. Mathematically, we can write
this linear relationship as follow \[ Y\approx \beta_0+\beta_1 X\]
\(\beta_0\) and \(\beta_1\) are unknown: \(\beta_0\) represents the
intercept and \(\beta_1\) the slope in the linear model. These values
are known as the model parameters or coefficients. Once we have used our
training data to produce estimates \(\widehat{\beta_0}\) and
\(\widehat{\beta}_1\) for the model coefficients, we can predict future
value \(\widehat{y}\) of \emph{Y} on the basisof a particular value,
\emph{x} of \emph{X} by computing
\[y= \widehat{\beta_0}+\widehat{\beta_1} x\] The main purpose is the
computation of the straigntline which is as close as possible of data
points. The must common approach involves the \emph{least square}
criterion. Let \(\widehat{y_i}=\widehat{\beta_0}+\widehat{\beta_1}x_i\)
be the prediction for \emph{Y} based on the \(i\)th value of \emph{X} .
Then \(e_i=y_i-\widehat{y_i}\) reprsents the \(i\)th residual; this is
the difference between the obseved response value and the \(i\)th
response value predicted by our linear model. The Residual Sum of
Squares (RSS) is defined as follow
\[RSS=\sum_{i=1}^{n}\big(y_i- \widehat{\beta_0}+\widehat{\beta_1} x_i\big)^²\]\(\beta_1\)

\end{document}
